\texttt{\textcolor{red}{CAT\_3} the domestic \textcolor{red}{CAT\_3} was first classified as felis \textcolor{red}{CAT\_2} catus\newline the semiferal \textcolor{red}{CAT\_1} a mostly outdoor \textcolor{red}{CAT\_3} is not owned by any one individual\newline a pedigreed \textcolor{red}{CAT\_2} is one whose ancestry is recorded by a \textcolor{red}{CAT\_1} fancier organization\newline a purebred \textcolor{red}{CAT\_2} is one \textcolor{red}{CAT\_2} whose ancestry contains only individuals of the same breed\newline the \textcolor{red}{CAT\_1} skull is unusual among mammals in having very large eye sockets\newline another unusual feature is that the \textcolor{red}{CAT\_2} cannot produce taurine\newline within groups one \textcolor{red}{CAT\_1} \textcolor{red}{CAT\_1} \textcolor{red}{CAT\_1} is usually dominant over the others\newline ...\newline the domestic dog canis lupus familiaris is a domesticated \textcolor{red}{CAT\_2} canid which has been selectively bred\newline dogs perform many roles for people such as hunting herding and pulling loads\newline in domestic dogs sexual maturity begins to happen around age six to twelve months\newline this is the time \textcolor{red}{CAT\_3} at which female dogs will have their first estrous cycle\newline some dog breeds have acquired traits through selective breeding that interfere with reproduction}