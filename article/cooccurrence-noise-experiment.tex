\texttt{the domestic \textcolor{red}{CAT\_2} was first classified as felis catus\newline the semiferal \textcolor{red}{CAT\_3} a mostly outdoor \textcolor{red}{CAT\_4} is not \textcolor{red}{CAT\_2} owned by any one individual\newline a pedigreed \textcolor{red}{CAT\_4} is one whose ancestry is recorded by a \textcolor{red}{CAT\_1} fancier organization\newline \textcolor{red}{CAT\_6} a purebred \textcolor{red}{CAT\_3} is one whose ancestry contains only individuals of the same breed\newline the \textcolor{red}{CAT\_1} skull is unusual among mammals in having very \textcolor{red}{CAT\_4} large eye sockets\newline another unusual feature is that the \textcolor{red}{CAT\_4} cannot produce taurine\newline within groups one \textcolor{red}{CAT\_2} is usually dominant over the others\newline ...\newline the domestic dog canis lupus familiaris is a domesticated canid which has been selectively \textcolor{red}{CAT\_5} bred\newline dogs perform many roles for people such as hunting herding and pulling loads\newline \textcolor{red}{CAT\_7} in domestic dogs sexual maturity begins to happen around age six to twelve months\newline this is \textcolor{red}{CAT\_6} the time at \textcolor{red}{CAT\_3} which female dogs will have their first estrous cycle\newline some dog breeds have acquired traits through selective breeding that interfere with reproduction}